\documentclass[a4paper, 12pt]{article}

\usepackage[T1]{fontenc}
\usepackage[utf8]{inputenc}
\usepackage[brazilian]{babel}
\usepackage[top = 2cm, bottom = 2cm, left = 2.5cm, right = 2.5cm]{geometry}

\title{\textbf{Exercícios de fixação - Curso de LaTeX}}
\author{Autor: Rodrigo Camara Barboza}
\date{Data: 20 de Junho de 2021}

\begin{document}
\maketitle

Abaixo, seguem os exercícios resolvidos:

%Exercício 1
\begin{equation}
\sqrt[3]{\left( \frac{2^{3} + 2^{5}}{10}\right)}
\end{equation}

%Exercício 2
\begin{equation}
\overline{(x \cdot y)^{4}} = \overline{x^{4}} \cdot \overline{y^{4}}
\end{equation}

%Exercício 3
\begin{equation}
\frac{a}{\sin\widehat{A}} = \frac{b}{\sin\widehat{B}} = \frac{c}{\sin\widehat{C}} = 2r
\end{equation}

%Exercício 4
\begin{equation}
\Vert \vec{u} \times \vec{v} \Vert = \Vert \vec{u} \Vert \cdot \Vert \vec{v} \Vert \cdot \sin(\theta)
\end{equation}

%Exercício 5
\begin{equation}
\frac{1}{\left(\frac{2}{3} \, cm/s\right)^{2}} \frac{\partial^{2}\Psi}{\partial t^{2}} - \frac{\partial^{2}\Psi}{\partial x^2}
\end{equation}

\end{document}